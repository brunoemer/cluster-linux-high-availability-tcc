\chapter{Virtualização}
O conceito virtualização surgiu na década de 60, sendo que um dos principais motivos foi a necessidade de um grande servidor,
conhecido como \textit{mainframe}, executar uma variedade de \textit{softwares}. Isso ocorreu pois cada \textit{mainframe} possuía
um próprio sistema operacional necessitando assim a criação de máquinas virtuais, mais conhecida como \textit{virtual machine} (VM).

Segundo \cite{carissimi2008}  falar sobre vantagens virtualizacao

Virtualização defini-se como uma camada entre o \textit{hardware} e o sistema operacional assim possibilitando a divisão e proteção
dos recursos físicos \cite{smithenair2005}. 

%Virtualização_ da teoria a soluções.pdf
%virtualizacao mainframes: pag 1 - Alta Disponibilidade em Servidores Virtualizados.pdf
%pag 32 - Rejuvenescimento e migracao de vms - Matheus D'Eça de Melo.pdf
%pag 20 - Implementação de Alta disponibilidade em Máquinas Virtuais utilizando Software Livre.pdf
%ferramentas pag 4 - Main - Virtualização de serviços baseado em contêineres -WandersonReis.pdf
%Com a arquitetura baseada em serviços, clientes podem utilizar serviços alocados em
%nuvem através dos web browsers. Unindo a computação em nuvem e a arquitetura baseada
%em serviços, aplicativos e demais recursos de TI podem ser oferecidos remotamente, como se
%estivessem localizados localmente. Vale a pena salientar que a arquitetura baseada em serviços
%permite monitorar em tempo real o uso dos recursos disponibilizados, colaborando assim para
%uma gerência mais eficiente a todo o sistema (LINTHICUM, 2009). Além disso, dependendo
%das interfaces definidas, pode-se ampliar consideravelmente o número de potenciais clientes.
%Recursos disponibilizados via navegador web, por exemplo, podem ser acessados tanto por
%computadores de mesa quanto por smartphones ou tablets.
%

%ver failover e failback em cluster?
