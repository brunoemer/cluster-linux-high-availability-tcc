\chapter{Virtualização}
\label{cap:virtualizacao}

O conceito virtualização surgiu na década de 60, sendo que muitas vezes havia a necessidade de cada usuário utilizar um ambiente individual 
com suas próprias aplicações e totalmente isolado dos outros usuários. Sabendo que naquela época era utilizado grandes servidores, conhecidos
como \textit{mainframes}, para a execução dessas aplicações, assim aproveitando ineficientemente o \textit{hardware}. Por esse motivo houve
a necessidade da criação de máquinas virtuais, mais conhecida como \ac{VM}, que teve forte expansão com um dos principais sistemas comerciais 
com suporte a virtualização, sistema operacional \textit{370} que foi desenvolvido pela \textit{IBM}. Na década de 80, houve uma redução da
utilização da virtualização devido a popularização de \textit{hardware} barato como o \ac{PC}. Na época era mais vantajoso colocar um \ac{PC} 
para cada usuário, do que investir em caros e complexos \textit{mainframes}. Devido ao crescente avanço e melhor desempenho do \ac{PC} e
ao surgimento da linguagem \textit{Java}, no início da década de 90, a tecnologia de virtualização retornou com o conceito de virtualização
de aplicação \cite{laureano2008}.
%O conceito virtualização surgiu na década de 60, sendo que um dos principais motivos foi a necessidade de um grande servidor,
%conhecido como \textit{mainframe}, executar uma variedade de \textit{softwares}. Isso ocorreu pois cada \textit{mainframe} necessitava
%do próprio sistema operacional, pois cada \textit{software} possuia além da aplicação todo o ambiente operacional no qual executava. Assim
%sendo necessário a criação de máquinas virtuais, mais conhecida como \ac{VM} \cite{carissimi2008}.

A virtualização foi definida nos anos 60 e 70 como uma camada entre o \textit{hardware} e o sistema operacional que possibilitava a 
divisão e proteção dos recursos físicos. Porém, atualmente ela engloba outros conceitos, como por exemplo a \ac{JVM}, que não virtualiza
um \textit{hardware}. Atualmente pode-se definir virtualização como uma camada de \textit{software} que utiliza os serviços fornecidos
de uma determinada interface de sistema para criar outra interface de mesmo nível, com isso poderá suprir a necessidade dos componentes do 
sistema que irão utilizá-la. Essa camada irá permitir a comunicação entre interfaces distintas, de forma que uma aplicação desenvolvida para
uma plataforma \textit{X} possa também executar em uma plataforma \textit{Y} \cite{laureano2008}.

Um ambiente de virtualização é composto basicamente por três componentes:
\begin{itemize}
 \item Sistema real: também pode ser chamado de hospedeiro, que é o \textit{hardware} onde o sistema de virtualização irá executar;
 \item Camada de virtualização: é conhecida como hipervisor ou também chamado de \ac{VMM}, tem como função criar interfaces virtuais a
 partir de interfaces físicas para a comunicação do sistema real com o sistema virtual;
 \item Sistema virtual: também conhecido como \textit{guest}, ou sistema convidado, que executa sobre o sistema real. Geralmente
 existem vários sistemas virtuais executando simultaneamente sobre o sistema real.
\end{itemize}

Máquinas virtuais podem ser divididas em dois grupos: máquinas virtuais de aplicação e máquinas virtuais de sistema. A primeira faz a
virtualização de uma aplicação, suporta apenas um processo ou aplicação. Um exemplo de máquina virtual de aplicação é a \ac{JVM}. 
A máquina virtual de sistema faz o suporte de sistemas operacionais convidados, com aplicações executando sobre ele, um exemplo é 
\textit{VMware} \cite{laureano2008}.
Virtualização de sistema utiliza abstração em sua arquitetura, por exemplo, ela transforma um disco físico em dois discos 
virtuais menores, sendo que esses discos virtuais são arquivos armazenados no disco físico. Sabendo que arquivos são uma abstração
em um disco físico, pode-se dizer que virtualização não é apenas uma camada de abstração do \textit{hardware}, ela faz a reprodução 
do \textit{hardware} \cite{smithenair2005}.
Existem várias formas de implementar a virtualização, portanto elas serão detalhas na Seção \ref{section:tiposvirt}.

\section{Vantagens}
\label{section:vantagensvirt}

Em muitos casos empresas utilizam serviços distribuídos entre servidores físicos, como, por exemplo, servidores de e-mail, hospedagens e 
banco de dados, com isso existe uma ociosidade grande de recursos. Portanto uma das grandes vantagens da virtualização é um melhor 
aproveitamento destes recursos, alocando vários serviços em um único servidor gerando um melhor aproveitamento do \textit{hardware} 
\cite{moreira2006}. Além disso, pode-se ter uma redução de custos com a administração e a manutenção dos servidores. Em um ambiente 
heterogênio pode-se também utilizar virtualização, pois ela permite a instalação de diversos sistemas operacionais em um único servidor.

Uma outra motivação para a utilização de virtualização consiste no custo da energia elétrica. A economia de energia pode ser obtida através 
da implantação de servidores mais robustos para substituir dezenas de servidores comuns. Outros fatores como refrigeração do ambiente e 
espaço físico utilizado também podem ser reduzidos com a implantação de virtualização de servidores, e consequentemente, reduzem os 
custos de energia também.

A virtualização favorece a implementação do conceito um servidor por serviço, que consiste em ter um servidor para cada serviço.
Mas porque não colocar todos serviços em um único servidor? Muitas vezes com uma variedade de serviços é necessário diferentes 
sistemas operacionais, ou os serviços necessitam rodar nas mesmas portas, portanto isto se torna inviável. Outro fator relevante que 
também favorece a implementação de um servidor por serviço é, caso exista uma falha de segurança em apenas um serviço, essa 
vulnerabilidade poderá comprometer todos os outros serviços 
\cite{carissimi2008}.

\section{Tipos de virtualização}
\label{section:tiposvirt}

\begin{itemize}
 \item Emulação
 \item Virtualização completa
 \item Paravirtualização
 \item Virtualização baseada em contêineres
\end{itemize}

ver jvm e wine

\section{Funcionamento}
\label{section:funcionamentovirt}

** Anéis de privilégios, compatibilidade, e hypervisor \cite{goncalves2009}.


%Virtualização_ da teoria a soluções.pdf
%virtualizacao mainframes: pag 1 - Alta Disponibilidade em Servidores Virtualizados.pdf
%pag 32 - Rejuvenescimento e migracao de vms - Matheus D'Eça de Melo.pdf
%pag 20 - Implementação de Alta disponibilidade em Máquinas Virtuais utilizando Software Livre.pdf
%ferramentas pag 4 - Main - Virtualização de serviços baseado em contêineres -WandersonReis.pdf

%Virtualizacao da teoria a solucoes:
%No entanto, essa abordagem trouxe como contra-partida a filosofia “um servidor
%por serviço”. Rapidamente, os responsáveis pelas áreas de TI se deram conta do
%problema (e custo) em gerenciar diferentes máquinas físicas, mesmo que tivessem o
%mesmo sistema operacional. Além disso, há problemas relacionados com consumo de
%energia elétrica, refrigeração, espaço físico, segurança física, etc. Nesse contexto, a
%virtualização surge como uma possibilidade de agregar os benefícios da componetização
%de sofware com a redução dos custos de manutenção de hardware e software. Assim, é
%possível manter a idéia de um “um servidor por serviço” sem ter um hardware
%específico.
%Essa abordagem é reforçada pela lei de Zipf [Adamic, 2008] que pode ser
%sintetizada da seguinte forma: a freqüência de um evento é proporcional a x −α , onde x é
%um ranking de comparação de um evento a outro. Alguns estudos [Breslau, Cao, Fan,
%Philips e Shenker, 2008] mostraram que a freqüência de acesso a servidores web e
%outros serviços Internet seguem uma distribuição Zipfian, o que, na prática, se traduz
%pelo fato de que a maioria dos acessos a serviços Internet é para uma minoria deles.
%Portanto, conclui-se que uma minoria de serviços está ativa enquanto a maioria está
%bloqueada a espera de requisições, o que, claramente, representa um desperdício de
%recursos. Para exemplificar, imagine, entre outros, o uso dos servidores de autenticação,
%DHCP, impressão, arquivos, e-mail, web e DNS em sua rede corporativa.

%Rejuvenescimento e migracao de vms:
%Com a arquitetura baseada em serviços, clientes podem utilizar serviços alocados em
%nuvem através dos web browsers. Unindo a computação em nuvem e a arquitetura baseada
%em serviços, aplicativos e demais recursos de TI podem ser oferecidos remotamente, como se
%estivessem localizados localmente. Vale a pena salientar que a arquitetura baseada em serviços
%permite monitorar em tempo real o uso dos recursos disponibilizados, colaborando assim para
%uma gerência mais eficiente a todo o sistema (LINTHICUM, 2009). Além disso, dependendo
%das interfaces definidas, pode-se ampliar consideravelmente o número de potenciais clientes.
%Recursos disponibilizados via navegador web, por exemplo, podem ser acessados tanto por
%computadores de mesa quanto por smartphones ou tablets.
%

%ver failover e failback em cluster?
