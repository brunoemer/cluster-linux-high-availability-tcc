\chapter{Conclusão}
\label{cap:conclusao}

Neste trabalho foi feito um estudo sobre uma empresa prestadora de serviços para Internet, analisando sua estrutura física e os seus servidores, 
bem como um detalhamento de todos serviços fornecidos pela empresa. Durante este estudo foram definidos os serviços críticos para a empresa.
Essa definição foi baseada em alguns critérios que permitem distinguir esses serviços.

Para atingir o objetivo deste trabalho, estudou-se os conceitos básicos de alta disponibilidade. Sabendo que o ambiente atual da empresa possui 
grande parte dos serviços em máquinas virtuais, foram estudados os principais conceitos sobre virtualização.

Também foi feito uma pesquisa de ferramentas para implementação de alta disponibilidade em um ambiente de virtualização. Criou-se uma proposta 
para tal implementação, sendo que esta é composta por um \textit{cluster} o qual é constituído por dois \textit{softwares}. Esses são o 
\textit{software} de gerenciamento do \textit{cluster} e o \textit{software} de replicação de dados. %nome dos softwares?
Esse \textit{cluster} será composto por dois servidores executando as máquinas virtuais contendo os serviços críticos. Esses servidores 
irão caracterizar uma redundância de \textit{software} sendo que caso um servidor falhe, as máquinas virtuais serão transferidas para o outro
servidor, assim possibilitando a disponibilidade dos serviços críticos.

Sendo assim, pode-se concluir que é possível criar um \textit{cluster} de alta disponibilidade com ferramentas de código aberto, e além disso 
pode-se utilizar virtualização em conjunto com essas ferramentas, para que, deste modo o ambiente se torne mais flexível e disponível. 

Entretanto, como neste trabalho ainda será feito a implementação em si, não pode-se afirmar que a solução irá funcionar corretamente
na empresa, devido a fatores como versão dos \textit{softwares} de virtualização, desempenho das ferramentas do ambiente de alta disponibilidade, 
versão do sistema operacional, entre outros.

\newpage
\section{Cronograma}
\label{section:cronograma}

A implementação fará parte da segunda etapa deste trabalho, bem como a elaboração da solução definitiva para o ambiente de alta disponibilidade. 
As tarefas da segunda etapa deste trabalho estão detalhadas na Tabela \ref{tab:impltarefas}, juntamente com as tarefas efetuadas até o momento. 
Para uma melhor organização, criou-se um cronograma com a distribuição das tarefas ao longo do próximo semestre. O cronograma atualizado 
deste semestre, juntamente com as tarefas do próximo semestre, encontra-se representado pela Tabela \ref{tab:implcronograma}.

\begin{table}[h!]\normalsize
\caption {Tarefas deste trabalho}
\label{tab:impltarefas}
\begin{center}
\begin{tabular}{|l|l|}\hline
 & \textbf{Descrição} \\\hline
1 & Redação TCC I \\\hline
2 & Estudo bibliográfico \\\hline
3 & Definição das ferramentas \\\hline
4 & Análise na empresa \\\hline
5 & Definição dos serviços críticos \\\hline
6 & Apresentação TCC I \\\hline
7 & Redação do TCC II \\\hline
8 & Elaboração da solução \\\hline
9 & Reorganização do ambiente de virtualização \\\hline
10 & Realização de testes \\\hline
11 & Implementação da solução \\\hline
12 & Análise dos resultados e medições \\\hline
13 & Apresentação do TCC II \\\hline
\end{tabular}
\end{center}
\end{table}

\begin{table}[h!]\normalsize % fonte tamanho normal
\caption {Cronograma deste trabalho}
\label{tab:implcronograma}
\begin{center}
\def\arraystretch{1}
\setlength{\tabcolsep}{0.15cm}
\begin{tabular}{|l|l|l|l|l|l|l|l|l|l|l|l|l|l|}\hline
 & \textbf{TCC} & \textbf{Jan} & \textbf{Fev} & \textbf{Mar} & \textbf{Abr} & \textbf{Mai} & \textbf{Jun} & \textbf{Jul} & \textbf{Ago} & \textbf{Set} & \textbf{Out} & \textbf{Nov} & \textbf{Dez} \\\hline
1 & TCC I & & & X & X & X & X & & & & & & \\\hline
2 & TCC I & X & X & X & X & X & & & & & & & \\\hline
3 & TCC I & & & & & X & X & & & & & & \\\hline
4 & TCC I & & & & & X & X & & & & & & \\\hline
5 & TCC I & & & & & X & X & & & & & & \\\hline
6 & TCC I & & & & & & & X & & & & & \\\hline
7 & TCC II & & & & & & & X & X & X & X & X & X \\\hline
8 & TCC II & & & & & & & X & X & X & & & \\\hline
9 & TCC II & & & & & & & X & X & & & & \\\hline
10 & TCC II & & & & & & & & X & X & X & & \\\hline
11 & TCC II & & & & & & & & & X & X & & \\\hline
12 & TCC II & & & & & & & & & & X & X & \\\hline
13 & TCC II & & & & & & & & & & & & X \\\hline
\end{tabular}
\end{center}
\end{table}
