\chapter{Conclusão}
\label{cap:conclusao}

Neste trabalho foi feito um estudo sobre uma empresa prestadora de serviços para Internet, analisando sua estrutura física e os seus servidores. 
Durante este estudo foram definidos os serviços críticos, para tanto considerou-se o impacto dos mesmos para a empresa através dos seguintes 
critérios: a quantidade de clientes e funcionários que utilizam o serviço; o número de requisições do serviço, sendo que esta pode ser requisições 
por segundo ou requisições simultâneas; e volume de elementos do serviço.
Destaca-se que foi necessário selecionar os serviços mais críticos para a empresa por não haver recursos necessários para implementar a
alta disponibilidade para todos os serviços.

Deste modo, os serviços críticos definidos foram: o serviço de \ac{DNS} recursivo, pois é utilizado para navegação de Internet de todos os clientes 
e funcionários do provedor; o serviço de autenticação \textit{Radius}, por influenciar diretamente na navegação dos clientes e armazenar dados 
importantes para o provedor; sistemas da empresa, esse serviço é importante pois a maior parte dos funcionários utilizam-no e também por ter um 
impacto indireto nos clientes; e serviço de telefonia interna, por ser responsável pela comunição tanto entre funcionários, como entre clientes 
e funcionários.

Após criou-se uma proposta de alta disponibilidade para esses serviços. Essa proposta é composta por um \textit{cluster} o qual é constituído 
por dois servidores físicos. Para o gerenciamento do \textit{cluster} será adotado o \textit{software} \textit{Ganeti}, que fará o gerenciamento, 
monitoramento e a transferência dos serviços para garantir a alta disponibilidade. Para a replicação de dados do \textit{cluster} foi adotado 
o \textit{software} \ac{DRBD}. Esse será responsável pela replicação dos dados dos dois servidores.

O ambiente de alta disponibilidade será composto por máquinas virtuais que executarão os serviços críticos. Para garantir a disponibilidade 
será utilizado a opção de migração em tempo real, fornecida pelo hipervisor \ac{KVM}, juntamente com o \textit{Ganeti}. Desta forma, 
caso um dos servidores falhe as máquinas virtuais serão migradas para o outro servidor. Destacando-se que bons resultados foram obtidos com os 
testes realizados com essas ferramentas.

Na próxima parte deste trabalho, primeiramente, será finalizada a elaboração da solução definitiva, além disso, será feito a organização do 
ambiente de virtualização da empresa, para então ser possível a implementação desta solução de alta disponibilidade. E por fim, será feito uma 
análise dos resultados desta implementação. O cronograma destas etapas está melhor detalhado na próxima seção.

% Sendo assim, pode-se concluir que é possível criar um \textit{cluster} de alta disponibilidade com ferramentas de código aberto, destacando que 
% pode-se utilizar virtualização em conjunto com essas ferramentas, para que, deste modo o ambiente se torne mais flexível e disponível. 
% Além disso, com esse estudo pode-se observar que existe uma variedade de ferramentas que possibilitam atingir o objetivo geral deste trabalho.

% Entretanto, como neste trabalho ainda será feito a implementação em si, não pode-se afirmar que a solução irá funcionar corretamente
% na empresa, devido a fatores como versão dos \textit{softwares} de virtualização, desempenho das ferramentas do ambiente de alta disponibilidade, 
% versão do sistema operacional, entre outros.

\section{Cronograma}
\label{section:cronograma}

Na Tabela \ref{tab:propcronograma} tem-se o cronograma correspondente a primeira parte deste trabalho. Como pode ser observado todas etapas foram
concluídas, sendo que houve uma modificação no item 3, definição das ferramentas.
Já na Tabela \ref{tab:implcronograma} tem-se o cronograma da segunda parte deste trabalho. Nesta parte será implementada a solução de 
alta disponibilidade e será realizados os testes.

\begin{table}[h!]\normalsize % fonte tamanho normal
\caption {Cronograma TCC I}
\label{tab:propcronograma}
\begin{center}
\def\arraystretch{1}
\setlength{\tabcolsep}{0.15cm}
\begin{tabular}{|l|l|l|l|l|l|l|l|l|l|l|l|l|l|}\hline
 & \textbf{Descrição} & \textbf{Jan} & \textbf{Fev} & \textbf{Mar} & \textbf{Abr} & \textbf{Mai} & \textbf{Jun} & \textbf{Jul} \\\hline
1 & Redação TCC I & & & X & X & X & X & X \\\hline
2 & Estudo bibliográfico & X & X & X & X & X & & \\\hline
3 & Definição das ferramentas & & & & & X & X & \\\hline
4 & Análise na empresa & & & & & X & X & \\\hline
5 & Definição dos serviços críticos & & & & & X & X & \\\hline
6 & Apresentação TCC I & & & & & & & X \\\hline
\end{tabular}
\end{center}
\end{table}

\begin{table}[h!]\normalsize % fonte tamanho normal
\caption {Cronograma TCC II}
\label{tab:implcronograma}
\begin{center}
\def\arraystretch{1}
\setlength{\tabcolsep}{0.15cm}
\begin{tabular}{|l|l|l|l|l|l|l|l|l|l|l|l|l|l|}\hline
 & \textbf{Descrição} & \textbf{Jul} & \textbf{Ago} & \textbf{Set} & \textbf{Out} & \textbf{Nov} & \textbf{Dez} \\\hline
1 & Redação do TCC II & X & X & X & X & X & X \\\hline
2 & Elaboração da solução & X & X & X & & & \\\hline
3 & Reorganização do ambiente de virtualização & X & X & & & & \\\hline
4 & Realização de testes & & X & X & X & & \\\hline
5 & Implementação da solução & & & X & X & & \\\hline
6 & Análise dos resultados e medições & & & & X & X & \\\hline
7 & Apresentação do TCC II & & & & & & X \\\hline
\end{tabular}
\end{center}
\end{table}
