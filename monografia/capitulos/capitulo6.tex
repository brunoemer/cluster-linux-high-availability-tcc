\chapter{Conclusão}
\label{cap:conclusao}

Neste trabalho foi feito um estudo de uma empresa prestadora de serviços para Internet, analisando sua estrutura física, bem como a configuração
de todos os servidores, além do detalhamento de todos serviços fornecidos pela empresa. Após isso, foi feito o levantamento dos serviços mais 
críticos para o funcionamento da empresa com base em alguns critérios relevantes para a empresa. Esse estudo foi feito com objetivo de encontrar 
uma solução de alta disponibilidade para o ambiente atual da empresa, para aumentar a disponibilidade e a confiabilidade dos serviços vitais desta 
empresa.

Para atingir tal objetivo, estudou-se os conceitos básicos de alta disponibilidade, bem como os conceitos de tolerância a falhas, redundância de
\textit{hardware} e \textit{software}. Como o ambiente atual da empresa possui grande parte dos serviços em máquinas virtuais, foi estudado
os principais conceitos sobre virtualização. Sendo que na virtualização, foi estudado os dois grupos de máquinas virtuais o qual inclui máquinas 
virtuais de aplicação e máquinas virtuais de sistema, além das arquiteturas e estratégias de implementação de máquinas virtuais. Também foram 
descritas as vantagens do uso da virtualização em alguns cenários distintos.

E por fim, pesquisou-se ferramentas para possibilitar a implementação de um ambiente tolerante a falhas com o uso de máquinas virtuais.
Sendo que, deu-se foco na busca por ferramentas de código aberto, para redução de custos. Para enfim propor uma solução de alta disponibilidade 
compatível com o ambiente atual da empresa, que é composto por servidores e ferramentas de código aberto, em sua maioria.

Sendo assim, pode-se concluir que é possível criar um \textit{cluster} de alta disponibilidade com ferramentas de código aberto, e além disso 
pode-se utilizar virtualização em conjunto com essas ferramentas, para que, deste modo o ambiente se torne mais flexível e disponível. 

Entretanto, como neste trabalho ainda será feito a implementação em si, não pode-se afirmar que a solução irá funcionar corretamente
na empresa, devido a fatores como versão dos \textit{softwares} de virtualização, desempenho das ferramentas do ambiente de alta disponibilidade, 
versão do sistema operacional, entre outros.

\newpage
\section{Cronograma}
\label{section:cronograma}

A implementação fará parte da segunda etapa deste trabalho, bem como a elaboração da solução definitiva para o ambiente de alta disponibilidade. 
Além disso, as tarefas da segunda etapa deste trabalho estão detalhadas na Tabela \ref{tab:impltarefas}. Para uma melhor organização, foi criado
um cronograma com a distribuição das tarefas ao longo do próximo semestre, esse cronograma está presente na Tabela \ref{tab:implcronograma}.

\begin{table}[h!]\normalsize
\caption {Tarefas do trabalho de conclusão II}
\label{tab:impltarefas}
\begin{center}
\begin{tabular}{|l|l|}\hline
 & Descrição \\\hline
1 & Redação do TCC II \\\hline
2 & Elaboração da solução \\\hline
3 & Reorganização do ambiente de virtualização \\\hline
4 & Realização de testes \\\hline
5 & Implementação da solução \\\hline
6 & Análise dos resultados e medições \\\hline
7 & Apresentação do TCC II \\\hline
\end{tabular}
\end{center}
\end{table}

\begin{table}[h!]\normalsize
\caption {Cronograma do trabalho de conclusão II}
\label{tab:implcronograma}
\begin{center}
\begin{tabular}{|l|l|l|l|l|l|l|}\hline
 & Jul & Ago & Set & Out & Nov & Dez \\\hline
1 & X & X & X & X & X & X \\\hline
2 & X & X & X & & & \\\hline
3 & X & X & & & & \\\hline
4 & & X & X & X & & \\\hline
5 & & & X & X & & \\\hline
6 & & & & X & X & \\\hline
7 & & & & & & X \\\hline
\end{tabular}
\end{center}
\end{table}
