\chapter{Conclusão}
\label{cap:conclusao}

Neste trabalho foi feito um estudo sobre uma empresa prestadora de serviços para Internet, analisando sua estrutura física e os seus servidores. 
Durante este estudo foram definidos os serviços críticos, para tanto considerou-se o impacto dos mesmos para a empresa.

Após criou-se uma proposta de alta disponibilidade para esses serviços. Essa proposta é composta por um \textit{cluster} o qual é constituído 
por dois servidores e \textit{softwares} que são responsáveis pelo gerenciamento do \textit{cluster} e pela replicação de dados. O
\textit{software} de gerenciamento adotado, que foi o \textit{Ganeti}, fará o gerenciamento, monitoramento e a transferência dos serviços para 
garantir a alta disponibilidade. O \textit{software} de replicação de dados adotado foi o \ac{DRBD}, que replicará os dados entre 
os dois servidores.

O ambiente de alta disponibilidade é composto por máquinas virtuais que contém os serviços críticos, sendo assim, utilizou-se a opção de 
migração em tempo real, fornecida pelo hipervisor \ac{KVM}, juntamente com o \textit{Ganeti}. Desta forma, possibilitou-se a alta disponibilidade 
dos serviços caso um dos servidores falhe. Além disso, nos testes realizados com essas ferramentas pode-se afirmar que obteve-se bons resultados.

Com esse estudo, destaca-se que existe uma variedade de ferramentas que possibilitam criar um ambiente de alta disponibilidade, de forma a atingir 
o objetivo deste trabalho.

% Sendo assim, pode-se concluir que é possível criar um \textit{cluster} de alta disponibilidade com ferramentas de código aberto, destacando que 
% pode-se utilizar virtualização em conjunto com essas ferramentas, para que, deste modo o ambiente se torne mais flexível e disponível. 
% Além disso, com esse estudo pode-se observar que existe uma variedade de ferramentas que possibilitam atingir o objetivo geral deste trabalho.

% Entretanto, como neste trabalho ainda será feito a implementação em si, não pode-se afirmar que a solução irá funcionar corretamente
% na empresa, devido a fatores como versão dos \textit{softwares} de virtualização, desempenho das ferramentas do ambiente de alta disponibilidade, 
% versão do sistema operacional, entre outros.

\newpage
\section{Cronograma}
\label{section:cronograma}

Na Tabela \ref{tab:propcronograma} tem-se o cronograma correspondente a primeira parte deste trabalho. Como pode ser observado todas etapas foram
concluídas, sendo que houve uma modificação no item 3, definição das ferramentas.
Já na Tabela \ref{tab:implcronograma} tem-se o cronograma da segunda etapa deste trabalho. Nesta etapa será implementada a solução de 
alta disponibilidade e será realizados os testes.

\begin{table}[h!]\normalsize % fonte tamanho normal
\caption {Cronograma TCC I}
\label{tab:propcronograma}
\begin{center}
\def\arraystretch{1}
\setlength{\tabcolsep}{0.15cm}
\begin{tabular}{|l|l|l|l|l|l|l|l|l|l|l|l|l|l|}\hline
 & \textbf{Descrição} & \textbf{Jan} & \textbf{Fev} & \textbf{Mar} & \textbf{Abr} & \textbf{Mai} & \textbf{Jun} & \textbf{Jul} \\\hline
1 & Redação TCC I & & & X & X & X & X & X \\\hline
2 & Estudo bibliográfico & X & X & X & X & X & & \\\hline
3 & Definição das ferramentas & & & & & X & X & \\\hline
4 & Análise na empresa & & & & & X & X & \\\hline
5 & Definição dos serviços críticos & & & & & X & X & \\\hline
6 & Apresentação TCC I & & & & & & & X \\\hline
\end{tabular}
\end{center}
\end{table}

\begin{table}[h!]\normalsize % fonte tamanho normal
\caption {Cronograma TCC II}
\label{tab:implcronograma}
\begin{center}
\def\arraystretch{1}
\setlength{\tabcolsep}{0.15cm}
\begin{tabular}{|l|l|l|l|l|l|l|l|l|l|l|l|l|l|}\hline
 & \textbf{Descrição} & \textbf{Jul} & \textbf{Ago} & \textbf{Set} & \textbf{Out} & \textbf{Nov} & \textbf{Dez} \\\hline
1 & Redação do TCC II & X & X & X & X & X & X \\\hline
2 & Elaboração da solução & X & X & X & & & \\\hline
3 & Reorganização do ambiente de virtualização & X & X & & & & \\\hline
4 & Realização de testes & & X & X & X & & \\\hline
5 & Implementação da solução & & & X & X & & \\\hline
6 & Análise dos resultados e medições & & & & X & X & \\\hline
7 & Apresentação do TCC II & & & & & & X \\\hline
\end{tabular}
\end{center}
\end{table}
