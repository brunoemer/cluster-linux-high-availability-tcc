\chapter{Implementação e resultados}
\label{cap:implementacaoresultados}

Neste capítulo será detalhado a implementação, com a configuração do \ac{OS}, do ambiente virtualizado e das ferramentas que irão compôr o
\textit{cluster} de alta disponibilidade. Posteriormente, serão efetuados testes e medições de resultados.

Descrever o que nao deu certo no projeto com o ganeti?

\section{Topologia}



\section{Configuração do \ac{OS}}

Foi feita a instalação do sistema operacional \textit{Ubuntu 14.04 \ac{LTS}} nos dois servidores.

\section{Configuração de rede}

A primeira etapa para incluir as máquinas virtuais a uma rede é criando uma \textit{bridge}. 

\begin{lstlisting}[language=bash]
  $ apt-get install bridge-utils
\end{lstlisting}

\section{Configuração de disco}


\section{Configuração do ambiente virtualizado}


\section{Configuração do cluster}

