\chapter{Implementação e resultados}
\label{cap:implementacaoresultados}

Neste capítulo será detalhado a implementação, com a configuração do \ac{OS}, do ambiente virtualizado e das ferramentas que irão compôr o
\textit{cluster} de alta disponibilidade. Posteriormente, serão efetuados testes e medições de resultados.

Nas próximas seções será demonstrado a configuração das ferramentas do ambiente de alta disponibilidade, do ambiente de virtualização e 
também do sistema operacional.

Descrever o que nao deu certo no projeto com o ganeti??

\section{Topologia}

A estrutura física adotada está representada na figura ??.

figura?? %parecida com Figura 4.2: Modelo de estrutura fisica.

Pode-se observar os dois servidores ligados a um \textit{switch} ...

Na figura ?? tem-se a imagem dos servidores, o primeiro \textit{Brina} (Dell PowerEdge 2950) no topo da imagem, e o servidor mais abaixo 
\textit{Piova} (Dell PowerEdge R410).
%#fotos servidores, switch ??

A estrutura que representa a configuração dos \textit{softwares} pode ser observado na figura ??. Onde tem-se o \textit{software} de gerenciamento
do cluster, o \textit{software} de replicação de dados, o sistema de arquivos, as máquinas virtuais e os serviços.
%#imagem do projeto??

\section{Configuração do \ac{OS}}

Foi feita a instalação do sistema operacional \textit{Ubuntu 14.04 \ac{LTS}} nos dois servidores. A configuração feita foi a básica do sistema,
com nome do servidor, configuração de rede, localização e instalação do servidor \ac{SSH}.


\section{Configuração de rede}

Um requisito para incluir as máquinas virtuais a uma rede é criando uma \textit{bridge}. 

\begin{lstlisting}[language=bash]
  $ apt-get install bridge-utils
\end{lstlisting}

\section{Configuração de disco}


\section{Configuração do ambiente virtualizado}

chaves ssh root??

\section{Configuração do cluster}

