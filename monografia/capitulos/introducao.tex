\chapter{Introdução}
\label{cap:introducao}
O crescente avanço tecnológico e o desenvolvimento da Internet provocou um aumento significativo no número de aplicações ou serviços que 
dependem da infraestrutura de \ac{TI}. Além disso, percebe-se um aumento no número de operações \textit{on-line} que são realizadas, 
tanto por organizações públicas ou privadas, quanto por grande parte da população.

Desta forma, a sociedade está cada vez mais dependente da tecnologia, de computadores e de sistemas. De fato, pode-se observar 
sistemas computacionais desde em uma farmácia, até em uma grande indústria. Sendo assim, a estabilidade e a disponibilidade desses 
sistemas apresenta um grande impacto em nosso dia-a-dia, pois um grande número de atividades cotidianas dependem deles.

Uma interrupção imprevista em um ambiente computacional poderá causar um prejuízo financeiro para a empresa que fornece o serviço, 
além de interferir na vida das pessoas que dependem de forma direta ou indireta desse serviço. 
Essa interrupção terá maior relevância para as corporações cujo o serviço ou produto final é fornecido através da Internet, 
como por exemplo, o comércio eletrônico, \textit{websites}, sistemas corporativos, entre outros. 
Em um ambiente extremo, pode-se imaginar o caos e o possível risco de perda de vidas que ocorreria em caso de uma falha 
em um sistema de controle aéreo \cite{costa2009}.

Para essas empresas um plano de contingência é fundamental para garantir uma boa qualidade de serviço, além de otimizar o desempenho 
das atividades, e também para fazer uma prevenção de falhas e uma recuperação rápida caso essas ocorram \cite{costa2009}.
De fato, hoje em dia a confiança em um serviço é um grande diferencial para a empresa fornecedora desse, 
sendo que a alta disponibilidade é fundamental para atingir esse objetivo.

A alta disponibilidade consiste em manter um sistema disponível por meio da tolerância a falhas, isto é, utilizando mecanismos que fazem a 
detecção, mascaramento e a recuperação de falhas, sendo que esses mecanismos podem ser implementados a nível de \textit{software} ou de 
\textit{hardware} \cite{reis2009}. Para que um sistema seja altamente disponível ele deve ser tolerante a falhas, sendo que a tolerância
a falhas é, frequentemente, implementada utilizando redundância. No caso de uma falha em um dos componentes evita-se a interrupção do sistema,
uma vez que o sistema poderá continuar funcionando utilizando o outro componente \cite{batista2007}.

Neste trabalho foi realizado um estudo sobre a implementação de um sistema de alta disponibilidade em uma empresa prestadora de serviços para 
Internet. Essa empresa oferece serviços, como por exemplo hospedagens de sites, \textit{e-mail}, sistemas de gestão, \textit{e-mail marketing}, 
entre outros. A empresa possui aproximadamente 60 servidores físicos e virtuais, e aproximadamente 9000 clientes, 
sendo que em períodos de pico atende em torno de 1000 requisições por segundo. 

Anteriormente, a empresa possuía somente redundância nas conexões de acesso à Internet, refrigeração e energia, com \textit{nobreaks} e geradores. 
Porém, essa empresa não possuía nenhuma redundância nos serviços que estavam sendo executados nos servidores. Desta forma, caso ocorresse
uma falha de \textit{software} ou de \textit{hardware}, os serviços ficariam indisponíveis. Neste trabalho foi realizada uma análise dos 
serviços oferecidos pela empresa, sendo que mecanismos de alta disponibilidade foram desenvolvidos para os serviços mais críticos. 
Para a redução dos custos foram utilizadas ferramentas gratuitas e de código aberto.

\section{Objetivos}
A empresa estudada não apresentava nenhuma solução de alta disponibilidade para seus serviços críticos. Desta forma, neste trabalho 
foi desenvolvida uma solução de alta disponibilidade para esses serviços, sendo que essa solução foi baseada no uso de ferramentas de 
código aberto e de baixo custo. Para que o objetivo geral fosse atendido, os seguintes objetivos específicos foram realizados:

\begin{itemize}
\item Identificação dos serviços críticos a serem integrados ao ambiente de alta disponibilidade;
\item Definição das ferramentas a serem utilizadas para implementar tolerância a falhas;
\item Realização de testes para a validação do sistema de alta disponibilidade que foi desenvolvido.
\end{itemize}

\section{Estrutura do trabalho}
O trabalho foi estruturado em oito capítulos, que são:

\begin{itemize}
 \item Capítulo \ref{cap:introducao}: apresenta a introdução e objetivos do trabalho;
 \item Capítulo \ref{cap:altadisponibilidade}: apresenta o conceito de alta disponibilidade e conceitos relacionados;
 \item Capítulo \ref{cap:virtualizacao}: é apresentado um breve histórico da virtualização, bem como o conceito de máquinas virtuais e as 
 estratégias utilizadas para a implementação das mesmas;
 \item Capítulo \ref{cap:infraempresa}: descreve o ambiente inicial da empresa e os serviços que são fornecidos por esta;
 \item Capítulo \ref{cap:servicoscriticos}: neste capítulo são definidos os critérios para selecionar os serviços críticos da empresa;
 \item Capítulo \ref{cap:softwares}: neste capítulo é apresentado as ferramentas que compõem o ambiente de alta disponibilidade;
 \item Capítulo \ref{cap:implementacaoresultados}: este capítulo apresenta a solução de alta disponibilidade, sendo que esta foi desenvolvida 
 baseada na utilização de virtualização. Além disso, são descritos os testes realizados e feita a análise dos resultados provenientes dos testes 
 no ambiente de alta disponibilidade;
 \item Capítulo \ref{cap:conclusao}: apresenta as conclusões do trabalho e as sugestões de trabalhos futuros.
\end{itemize}
