\chapter{Proposta de solução}
\label{cap:propostadesolucao}

Neste capítulo será proposto uma solução de implementação de um ambiente de alta disponibilidade para os serviços críticos da empresa, que 
foram selecionados no capítulo anterior. Com isso pretende-se atingir o objetivo deste trabalho.

O primeiro passo desta implementação é fazer uma reorganização das máquinas virtuais entre os servidores atuais, para então liberar o 
\textit{hardware} suficiente, assim possibilitando a implementação do ambiente de alta disponibilidade. 

As ferramentas necessárias para a implementação podem ser divididas em dois tipos: ferramenta de replicação de dados (Seção \ref{section:toolrepl}) 
e ferramenta que faz o monitoramento e a tranferências das máquinas virtuais em caso de falhas (Seção \ref{section:toolcluster}).

\section{Ferramentas de replicação de dados}
\label{section:toolrepl}

Replicação de dados pode ser feita de diversas maneiras, pode ser a nível de aplicação ou até mesmo a nível de \textit{hardware}.
Dependendo do objetivo e da aplicação pode-se usar ferramentas como por exemplo o \textit{rsync}, que faz o sincronismo de dados de uma origem
para um destino. Sendo assim, não é possível utilizar essa ferramenta, pois ela não faz a replicação em tempo real, ou seja, se for necessário
utilizar os dados de destino ocorrerá perda de dados. Outra forma de replicação é a de discos com \ac{RAID} por exemplo, essa solução é eficaz
para garantir que o sistema não fique indisponível em caso de falha de discos\footnote{Lembrando que essa solução é utilizada no ambiente atual
para aumentar a disponibilidade dos servidores}, porém não garante a disponibilidade quando algum componente de \textit{hardware} falhar 
\cite{zaminhani2008}.

A solução de replicação ideal para esta implementação é um espelhamento de dados através da redes, assim permitindo a cópia dos dados para uma
máquina remota. Essa solução além de fazer a replicação dos dados, faz a redundância de todo \textit{hardware}.

A ferramenta escolhida para replicação de dados na solução de alta disponibilidade desse trabalho foi o \ac{DRBD}. Essa ferramenta é de código
aberto, e permite a replicação de dados em tempo real de um dispositivo local. 
APROFUNDAR AQUI OU NA IMPLEMENTAÇÂO?
% primario e secundario zaminhani2008

\section{Ferramentas de gerenciamento de cluster}
\label{section:toolcluster}

... \textit{Heartbeat} \cite{zaminhani2008}.


%reorganização de vms
%ferramentas selecionadas, colocar motivo para escolher e citar ferramentas parecidas
%muitos servicos, melhor solucao utilizar 2 servidores para fazer redundancia
%em caso de falha de um servidor fisico...
%ferramentas open source...
%colocar a disponibilidade do nagios do ano passado?
