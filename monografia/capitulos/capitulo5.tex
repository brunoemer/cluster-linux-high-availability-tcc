\chapter{Proposta de solução}
\label{cap:propostadesolucao}

Neste capítulo será proposto uma solução de implementação de um ambiente de alta disponibilidade para os serviços críticos da empresa, que 
foram selecionados no capítulo anterior...

O primeiro passo para essa implementação é fazer uma reorganização das máquinas virtuais, para possibilitar a liberação de \textit{hardware} 
suficiente para montar o ambiente de alta disponibilidade. 

As ferramentas necessárias para a implementação podem ser divididas em dois tipos: ferramenta de replicação de dados e ferramenta que faz o 
monitoramento e a tranferências das máquinas virtuais em caso de falhas.



%reorganização de vms
%ferramentas selecionadas, colocar motivo para escolher e citar ferramentas parecidas
%muitos servicos, melhor solucao utilizar 2 servidores para fazer redundancia
%em caso de falha de um servidor fisico...
%ferramentas open source...
%colocar a disponibilidade do nagios do ano passado?
