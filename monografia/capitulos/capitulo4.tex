\chapter{Estudo de caso ? esse titulo ?}
\label{cap:estudodecaso}

Após terem sido compreendidos os conceitos de alta disponibilidade e de virtualização, pode-se iniciar a análise da estrutura atual da empresa.
Essa empresa, que fornece serviços de hospedagens, será o foco desse trabalho, sabendo que ela atualmente possui redundância de refrigeração 
e energia. Assim, essas redundâncias tem como objetivo prover um ambiente físico estável e confiável.

Para ser possível propor uma solução de alta diponibilidade é necessário conhecer o cenário atual da empresa. 
Nas próximas seções serão detalhados: o ambiente físico (Seção \ref{section:estfis}); a estrutura lógica, com a relação de servidores físicos
e virtuais (Seção \ref{section:estlog}); todos os serviços fornecidos pela empresa (Seção \ref{section:serv}); e por fim o levantamento
dos serviços críticos (Seção \ref{section:servcrit}).

\section{Estrutura física}
\label{section:estfis}

A estrutura atual da empresa é composta por quatorze servidores montados em um \textit{rack}. A configuração desses servidores esta listado
na Tabela \ref{tab:servfisicos}.

\begin{table}
\caption {Configuração dos servidores físicos}
\label{tab:servfisicos}
\begin{center}
\begin{tabular}{|p{4.3cm}|p{3cm}|p{3cm}|l|}\hline
Processador & Memória & Disco & Marca / modelo\\\hline
Intel Core 2 Duo \newline CPU E6750 2.66GHz & 2GB DDR2 & 5,5TB SATA & \\\hline
2 x Intel Xeon \newline CPU E5410 2.33GHz & 24GB DDR2 & 6 x 300GB SAS & Dell PowerEdge 2950\\\hline
2 x Intel Xeon \newline CPU E5310 1.60GHz & 12GB DDR2 & 2 x 73GB SAS & Dell PowerEdge 2950\\\hline
2 x Intel Xeon \newline CPU 3.20GHz & 4GB DDR2 & 2 x 146GB SCSI & Dell PowerEdge 1850\\\hline
1 x Intel Xeon \newline CPU E5-2650 2.00GHz & 32GB DDR3 & 6 x 2TB SATA & IBM System x3650 M4\\\hline
Intel Core 2 Quad \newline CPU Q9550 2.83GHz & 4GB DDR2 & 120GB SSD & \\\hline
Intel Core 2 Duo \newline CPU E4500 2.20GHz & 4GB DDR2 & 500GB SATA & \\\hline
2 x Intel Xeon \newline CPU E5530 2.40GHz & 32GB DDR3 & 4 x 500G SATA & Dell PowerEdge R410\\\hline
2 x  Intel Xeon \newline CPU E5630 2.53GHz & 32GB DDR3 & 4 x 300GB SAS & HP ProLiant DL360 G7\\\hline
Intel Xeon \newline CPU X3330 2.66GHz & 8GB DDR2 & 750GB SATA & \\\hline
2 x Intel Xeon \newline CPU E5-2620 2.00GHz & 32GB DDR3 & 5 x 1TB SATA & Dell PowerEdge R620\\\hline
2 x Intel Xeon \newline CPU E5649 2.53GHz & 32GB DDR3 & 6 x 300GB SAS \newline 2 x 146GB SAS & HP ProLiant DL380 G7\\\hline
Intel Xeon CPU \newline E3-1220 3.10GHz & 16GB DDR3 & 2 x 3TB SATA & Dell PowerEdge R210 II\\\hline
Intel Pentium Dual CPU E2180 2.00GHz & 4GB DDR2 & 2,5TB SATA & \\\hline
\end{tabular}
\end{center}
\end{table}

diagrama físico

graficos cpu memoria disco

\section{Estrutura lógica}
\label{section:estlog}

Atualmente sete servidores são utilizados para virtualização.

diagramas máquinas de virtualização

\section{Serviços}
\label{section:serv}

A empresa fornece serviços diversos, desde hospedagens de sites até \textit{DNS} recursivo para um provedor de internet.
A Tabela \ref{tab:servporservidor} mostra todos os servidores, incluindo virtuais, e seus respectivos serviços.

\begin{table}
\caption {Serviços por servidor}
\label{tab:servporservidor}
\begin{center}
\begin{tabular}[]{|l|l|l|}\hline
Servidores & Serviços & Descrição\\\hline
asp & HTTP/ASP & Servidor web linguagem ASP\\\hline
backup & HTTP+SSH & Servidor de backup equipamentos de rede do provedor\\\hline
bello & BACULA-STORAGE & Servidor de storage do bacula, para backup dos outros servidores\\\hline
brina & VIRTUALIZACAO & Servidor de virtualização\\\hline
cacti & CACTI & \\\hline
dati & MYSQL & \\\hline
dio & HTTP/PHP4 & \\\hline
esibire & VIDEO STREAMING & \\\hline
fatefurbo & GERENCIA FIBRA & \\\hline
fiberhome & GERENCIA FIBRA & \\\hline
fulmine & VIRTUALIZACAO & \\\hline
hotspot &  & \\\hline
ledriovardar &  & \\\hline
masterauth &  & \\\hline
merak &  & \\\hline
miatanto &  & \\\hline
mondoperso &  & \\\hline
monete &  & \\\hline
monit &  & \\\hline
nino &  & \\\hline
ns &  & \\\hline
ottico &  & \\\hline
parla &  & \\\hline
passata &  & \\\hline
passata2 &  & \\\hline
piova & VIRTUALIZACAO & \\\hline
pomodoro &  & \\\hline
postfix &  & \\\hline
quebei &  & \\\hline
raggio & VIRTUALIZACAO & \\\hline
rauco &  & \\\hline
roncon &  & \\\hline
ronconradius &  & \\\hline
servo &  & \\\hline
servo6 &  & \\\hline
sfrunhon &  & \\\hline
simplesip &  & \\\hline
soldi &  & \\\hline
speedauth &  & \\\hline
tempesta & VIRTUALIZACAO & \\\hline
trapel &  & \\\hline
tuono & VIRTUALIZACAO & \\\hline
venti & VIRTUALIZACAO & \\\hline
vigilante &  & \\\hline
vinicolagaribaldi &  & \\\hline
\end{tabular}
\end{center}
\end{table}


\section{Serviços críticos}
\label{section:servcrit}


