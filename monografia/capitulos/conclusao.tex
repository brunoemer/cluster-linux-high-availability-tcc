\chapter{Conclusão}
\label{cap:conclusao}

Neste trabalho foi feito um estudo sobre uma empresa prestadora de serviços para Internet, analisando sua estrutura física e os serviços oferecidos 
por esta. Durante este estudo foram definidos os serviços críticos. Para tanto considerou-se o impacto dos mesmos para a empresa, medido através
da quantidade de clientes e funcionários que utilizam o serviço. Destaca-se que foi necessário selecionar os serviços mais críticos, por não 
haver recursos necessários para implementar a alta disponibilidade para todos os serviços.

Deste modo, os serviços críticos definidos foram: o serviço de \ac{DNS} recursivo, pois é utilizado para navegação de Internet de todos os clientes 
e funcionários do provedor; o serviço de autenticação \textit{Radius}, por influenciar diretamente na navegação dos clientes e armazenar dados 
importantes para o provedor; sistemas da empresa, uma vez que todos os funcionários utilizam-nos e também por ter um impacto indireto nos 
clientes; e serviço de telefonia interna, por ser responsável pela comunicação tanto entre funcionários, como entre clientes e funcionários.

Após criou-se uma proposta de alta disponibilidade para esses serviços. Essa proposta é composta por um \textit{cluster} o qual é constituído 
por dois servidores físicos. Para o gerenciamento do \textit{cluster} foi adotado o \textit{software} \textit{Pacemaker}, que é responsável pelo 
gerenciamento, monitoramento e a transferência das máquinas virtuais, as quais executam os serviços. Para a replicação de dados 
do \textit{cluster} foi adotado o \textit{software} \ac{DRBD}. Esse é responsável pela replicação dos dados dos dois servidores.

O ambiente de alta disponibilidade é composto por máquinas virtuais que executam os serviços críticos. Para garantir a disponibilidade 
foi utilizada a opção de migração em tempo real, fornecida pelo hipervisor \ac{KVM}, juntamente com o \textit{Pacemaker}. Desta forma, 
caso um dos servidores falhe as máquinas virtuais serão migradas para o outro servidor.

Ao final do trabalho, criou-se uma metodologia de testes para validar a solução de alta disponibilidade. Esses testes simulam falhas de 
\textit{hardware}, de energia elétrica, de \textit{software} e manutenções previamente agendadas foram implementados e aplicados, com isso
foi comprovado que a disponibilidade dos serviços críticos foi aumentada.
Desta forma, é possível concluir que é possível aumentar a disponibilidade de serviços em máquinas virtuais utilizando uma solução de
código aberto e de baixo custo.

\newpage

\section{Trabalho futuros}
\label{section:trabalhosfuturos}

Com a implementação apresentada neste trabalho pode-se perceber a variedade de ferramentas existentes para implantar soluções de alta 
disponibilidade. Deste modo, pode-se testar outras ferramentas de alta disponibilidade.
Como proposta de continuidade deste trabalho pode-se ampliar o ambiente, através da inclusão de mais serviços e mais \textit{hardware} ao 
\textit{cluster}. Também pode-se configurar uma interface para o gerenciamento do \textit{software} \textit{Pacemaker}, pois ele é configurado 
atualmente apenas por linha de comando.

