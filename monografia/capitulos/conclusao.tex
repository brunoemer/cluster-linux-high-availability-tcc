\chapter{Conclusão}
\label{cap:conclusao}

Neste trabalho foi feito um estudo sobre uma empresa prestadora de serviços para Internet, analisando sua estrutura física e os serviços oferecidos 
por esta. Durante este estudo foram definidos os serviços críticos. Para tanto considerou-se o impacto dos mesmos para a empresa, medido através
da quantidade de clientes e funcionários que utilizam o serviço. Destaca-se que foi necessário selecionar os serviços mais críticos, por não 
haver recursos necessários para implementar a alta disponibilidade para todos os serviços.

Deste modo, os serviços críticos definidos foram: o serviço de \ac{DNS} recursivo, pois é utilizado para navegação de Internet de todos os clientes 
e funcionários do provedor; o serviço de autenticação \textit{Radius}, por influenciar diretamente na navegação dos clientes e armazenar dados 
importantes para o provedor; sistemas da empresa, uma vez que todos os funcionários utilizam-nos e também por ter um impacto indireto nos 
clientes; e serviço de telefonia interna, por ser responsável pela comunicação tanto entre funcionários, como entre clientes e funcionários.

Após criou-se uma proposta de alta disponibilidade para esses serviços. Essa proposta é composta por um \textit{cluster} o qual é constituído 
por dois servidores físicos. Para o gerenciamento do \textit{cluster} será adotado o \textit{software} \textit{Ganeti}, que fará o gerenciamento, 
monitoramento e a transferência das máquinas virtuais, que executarão os serviços, para garantir a alta disponibilidade. Para a replicação de dados 
do \textit{cluster} foi adotado o \textit{software} \ac{DRBD}. Esse será responsável pela replicação dos dados dos dois servidores.

O ambiente de alta disponibilidade será composto por máquinas virtuais que executarão os serviços críticos. Para garantir a disponibilidade 
será utilizada a opção de migração em tempo real, fornecida pelo hipervisor \ac{KVM}, juntamente com o \textit{Ganeti}. Desta forma, 
caso um dos servidores falhe as máquinas virtuais serão migradas para o outro servidor. Destaca-se que bons resultados foram obtidos com os 
testes realizados com ambas as ferramentas.

Na próxima etapa deste trabalho, será finalizada a elaboração da solução, além disso, será feita a organização do ambiente de virtualização 
da empresa, para então ser possível a implementação desta solução de alta disponibilidade. E por fim, será feita uma análise dos resultados 
desta implementação. O cronograma destas etapas está melhor detalhado na próxima seção.

% Sendo assim, pode-se concluir que é possível criar um \textit{cluster} de alta disponibilidade com ferramentas de código aberto, destacando que 
% pode-se utilizar virtualização em conjunto com essas ferramentas, para que, deste modo o ambiente se torne mais flexível e disponível. 
% Além disso, com esse estudo pode-se observar que existe uma variedade de ferramentas que possibilitam atingir o objetivo geral deste trabalho.

% Entretanto, como neste trabalho ainda será feito a implementação em si, não pode-se afirmar que a solução irá funcionar corretamente
% na empresa, devido a fatores como versão dos \textit{softwares} de virtualização, desempenho das ferramentas do ambiente de alta disponibilidade, 
% versão do sistema operacional, entre outros.

\newpage

\section{Trabalho futuros}
\label{section:trabalhosfuturos}



