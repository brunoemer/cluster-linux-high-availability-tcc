\chapter{Conclusão}
\label{cap:conclusao}

Neste trabalho foi feito um estudo sobre uma empresa prestadora de serviços para Internet, analisando sua estrutura física e os serviços oferecidos 
por essa. Durante este estudo foram definidos os serviços críticos. Para tanto considerou-se o impacto dos mesmos para a empresa, medido através
da quantidade de clientes e funcionários que utilizam o serviço. Destaca-se que foi necessário selecionar os serviços críticos, por não 
haver recursos suficientes para a implementação de uma solução de alta disponibilidade para todos os serviços.

Deste modo, os serviços críticos definidos foram: o serviço de \ac{DNS} recursivo, pois é utilizado para navegação de Internet de todos os clientes 
e funcionários do provedor; o serviço de autenticação \textit{Radius}, por influenciar diretamente na navegação dos clientes e armazenar dados 
importantes para o provedor; sistemas da empresa, uma vez que todos os funcionários os utilizam e também por ter um impacto indireto nos 
clientes; e serviço de telefonia interna, por ser responsável pela comunicação tanto entre funcionários, como entre clientes e funcionários.

Após implementou-se um ambiente de alta disponibilidade para esses serviços, sendo que esse ambiente é composto por um \textit{cluster} o qual é 
constituído por dois servidores físicos. Para o gerenciamento do \textit{cluster} foi adotado o \textit{software} \textit{Pacemaker}, que é 
responsável pelo monitoramento e a transferência das máquinas virtuais, as quais executam os serviços. Para a replicação de dados 
do \textit{cluster} foi adotado o \textit{software} \ac{DRBD}, que é responsável pela replicação dos dados entre os dois servidores.

O ambiente de alta disponibilidade é composto por máquinas virtuais que executam os serviços críticos. Para garantir a disponibilidade 
foi utilizada a opção de migração em tempo real, fornecida pelo hipervisor \ac{KVM}, juntamente com o \textit{Pacemaker}. Desta forma, caso seja 
necessário fazer uma manutenção em um dos servidores, as máquinas virtuais serão migradas para o outro servidor sem gerar indisponibilidade.
Além disso, caso um servidor falhe, as máquinas virtuais serão automaticamente iniciadas no outro servidor, diminuindo assim a indisponibilidade
dos serviços e o impacto para a empresa.

Os testes realizados mostraram que em casos de falhas de \textit{hardware}, de energia elétrica ou de \textit{software}, o sistema manteve-se
disponível, uma vez que os serviços foram migrados para um outro nó. Além disso, o ambiente de alta disponibilidade possibilitou
realizar manutenções previamente agendadas sem gerar indisponibilidade para os serviços.
Também foi possível analisar a comparação de disponibilidade feita entre o ambiente de alta disponibilidade implementado e o antigo ambiente,
e perceber que houve uma melhora na disponibilidade.

Desta forma, conclui-se que é possível aumentar a disponibilidade de serviços em máquinas virtuais utilizando uma solução 
de código aberto e de baixo custo.
Conclui-se também que em caso de uma falha permanente de um \textit{hardware}, os dados permanecerão íntegros e disponíveis, assim garantindo 
uma maior segurança dos dados e informações da empresa.

\section{Trabalho futuros}
\label{section:trabalhosfuturos}

Durante o estudo realizado nesse trabalho verificou-se uma grande variedade de \textit{softwares} que podem ser utilizados para a implementação 
de soluções de alta disponibilidade. Deste modo, sugere-se que seja feito um estudo de viabilidade do uso das ferramentas como \textit{GlusterFS} 
e \textit{Ceph}, em conjunto com uma ferramenta de gerenciamento de nuvem, como por exemplo, o \textit{OpenStack}, que trarão mais alguns 
benefícios além da alta disponibilidade.

Além disso, sugere-se a inclusão de todos os serviços da empresa nesse ambiente. Destaca-se que em um curto prazo, pretende-se fazer a inclusão 
de alguns serviços com nível de criticidade médio. Por fim, pode-se efetuar uma medição de disponibilidade por um período mais longo. A partir
dessa medição será possível avaliar melhor o ambiente desenvolvido, bem como obter resultados mais consistentes.
