
\chapter{}
\label{cap:apscripts}

\section{Script indisponibilidade}
\label{ap:scriptindisp}

\begin{lstlisting}[language=bash]
#!/bin/bash
# Copyright (c) 2008 Wanderson S. Reis <wasare@gmail.com>
#
# Adaptado para gerar estatisticas do ping.
# Bruno Emer <emerbruno@gmail.com> 2016
#
# parametros :
#endereco IP e numero de sequencia do teste para controle
# ( opcional )
#
IP=$1

ping $1 > $1-stat-$2.log &
PINGPID=$!

RETORNO=0
while [ $RETORNO -eq 0 ]
do
        ping -c 1 $1 2> /dev/null 1> /dev/null
        RETORNO=$?
done
INICIAL=`date +%s`
while [ $RETORNO -ne 0 ]
do
        ping -c 1 $1 2> /dev/null 1> /dev/null
        RETORNO=$?
done
FINAL=`date +%s`
INTERVALO=$(( $FINAL - $INICIAL ))
echo "host $1 : $INTERVALO s" > $1-downtime-$2.log

#pkill -SIGINT ping
kill -SIGINT $PINGPID

exit 0
\end{lstlisting}


\section{Script manutenção Pacemaker}
\label{ap:scriptmanutencao}

\begin{lstlisting}[language=bash]
#!/bin/bash

# ------------------------------------------------------------------
# Copyright (c) 2016-09-02 Bruno Emer <emerbruno@gmail.com>
#
# Verifica e faz reinicializacao de um node do cluster pacemaker / drbd
# Instalacao:
# Agendar no cron de cada node em dias diferentes
#
# ------------------------------------------------------------------

PROGNAME=`/usr/bin/basename $0`
NODE=`hostname`
ONLINE_CHECK_RES="OCFS_MOUNT_CLONE"
STANDBY_CHECK_RES="MS_DRBD"

logger "Script manutencao cluster $PROGNAME node: $NODE"

#verificacoes
#se cluster esta ok
NAG_CHECK=$(/usr/lib/nagios/plugins/check_crm_v0_7)
NAG=$?
logger "Cluster nagios check: $NAG_CHECK"
if [ $NAG -ne 0 ]; then
        logger "Erro. Cluster nao esta ok"
        exit
fi

# ------------------------------------------------------------------
#recovery - retorna node para online se ja foi reiniciado
if [ -f pacemaker_reboot.tmp ]; then
        #verifica se pacemaker esta iniciado
        service pacemaker status
        PACEMAKER_ST=$?
        if [ $PACEMAKER_ST -eq 0 ]; then
                logger "Online $NODE"
                crm node online $NODE
                rm pacemaker_reboot.tmp
        fi
else
# ------------------------------------------------------------------
#destroy - derruba node para reiniciar
        #se outro node esta online
        RES_CHECK=$(crm resource show $ONLINE_CHECK_RES)
        NODE_LIST=$(crm_node -l |awk '{print $2}' |grep -v $NODE)
        NODES_N=$(crm_node -l |awk '{print $2}' |grep -v $NODE |wc -l)
        NODES_ON=0
        for row in $NODE_LIST; do
                RES_ON=$(echo "$RES_CHECK" |grep "$row")
                if [ -n "$RES_ON" ]; then
                        logger "$row online"
                        ((NODES_ON++))
                fi
        done
        if [ $NODES_ON -lt $NODES_N ]; then
                logger "Erro. Algum node nao esta online"
                exit
        fi

        #desativa servicos do node
        logger "Standby $NODE"
        crm node standby $NODE

        #aguarda node ficar livre, vms down e drbd down
        while :; do
                RES_CHECK_DRBD=$(crm resource show $STANDBY_CHECK_RES)
                RES_DRBD_ON=$(echo "$RES_CHECK_DRBD" |grep "$NODE")
                VMS_NUM=$(virsh list --name |wc -l)
                if [ -z "$RES_DRBD_ON" ] && [ $VMS_NUM -le 1 ]; then
                        logger "Pronto para reiniciar"
                        break
                else
                        logger "Servicos ainda executando"
                fi
                sleep 30
        done

        #escreve arquivo para recuperar node depois do reboot
        echo $(date +"%Y-%m-%d") > pacemaker_reboot.tmp

        #reinicia node
        logger "Rebooting..."
        reboot

fi
\end{lstlisting}
