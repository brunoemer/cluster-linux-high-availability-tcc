
\chapter{}
\label{cap:apscripts}

\section{Script indisponibilidade}
\label{ap:scriptindisp}

\begin{lstlisting}[language=bash]
#!/bin/bash
# Copyright (c) 2008 Wanderson S. Reis <wasare@gmail.com>
#
# Adaptado para gerar estatísticas do ping.
# Bruno Emer <emerbruno@gmail.com> 2016
#
# parâmetros :
#endereço IP e número de sequência do teste para controle
# ( opcional )
#
IP=$1

ping $1 > $1-stat-$2.log &
PINGPID=$!

INITMP=`date +%s`
RETORNO=0
while [ $RETORNO -eq 0 ]
do
	ping -c 1 $1 2> /dev/null 1> /dev/null
	RETORNO=$?

	TIMECUR=$(( `date +%s` - $INITMP ))
        if [ $TIMECUR -gt 90 ]; then
                break;
        fi
done
INICIAL=`date +%s`
while [ $RETORNO -ne 0 ]
do
	ping -c 1 $1 2> /dev/null 1> /dev/null
	RETORNO=$?
done
FINAL=`date +%s`
INTERVALO=$(( $FINAL - $INICIAL ))
echo "host $1 : $INTERVALO s" > $1-downtime-$2.log

while :; do
	TIMECUR=$(( `date +%s` - $INITMP ))
	if [ $TIMECUR -gt 300 ]; then
                break;
        fi
done

#pkill -SIGINT ping
kill -SIGINT $PINGPID

exit 0
\end{lstlisting}


\section{Script manutenção Pacemaker}
\label{ap:scriptmanutencao}

\begin{lstlisting}[language=bash]
#!/bin/bash

# ------------------------------------------------------------------
# Copyright (c) 2016-09-02 Bruno Emer <emerbruno@gmail.com>
#
# Verifica e faz reinicialização de um node do cluster pacemaker / drbd
# Instalação:
# Agendar no cron de cada node em dias diferentes
#
# ------------------------------------------------------------------

PROGNAME=`/usr/bin/basename $0`
NODE=`hostname`
ONLINE_CHECK_RES="OCFS_MOUNT_CLONE"
STANDBY_CHECK_RES="MS_DRBD"

logger "Script manutencao cluster $PROGNAME node: $NODE"

# ------------------------------------------------------------------
#recovery - retorna node para online se já foi reiniciado
if [ -f pacemaker_reboot.tmp ]; then
	#verifica se pacemaker esta iniciado
	service pacemaker status
	PACEMAKER_ST=$?
	if [ $PACEMAKER_ST -ne 0 ]; then
		logger "Inicia pacemaker"
		service pacemaker start
	fi
	service pacemaker status
        PACEMAKER_ST=$?
        if [ $PACEMAKER_ST -eq 0 ]; then
		logger "Online $NODE"
                /usr/sbin/crm node online $NODE
                rm pacemaker_reboot.tmp
	fi
else
# ------------------------------------------------------------------
#destroy - derruba node para reiniciar
	#verificações
	#se cluster esta ok
	NAG_CHECK=$(/usr/lib/nagios/plugins/check_crm_v0_7)
	NAG=$?
	logger "Cluster nagios check: $NAG_CHECK"
	if [ $NAG -ne 0 ]; then
		logger "Erro. Cluster nao esta ok"
		exit
	fi

	#se já reiniciou não executa
	UPTIME=`awk '{print $1}' /proc/uptime`
	UPINT=${UPTIME%.*}
	if [ $UPINT -lt 86400 ]; then
		logger "Já reiniciado"
		exit
	fi

	#se outro node esta online
	RES_CHECK=$(/usr/sbin/crm resource show $ONLINE_CHECK_RES)
	NODE_LIST=$(/usr/sbin/crm_node -l |awk '{print $2}' |grep -v $NODE)
	NODES_N=$(/usr/sbin/crm_node -l |awk '{print $2}' |grep -v $NODE |wc -l)
	NODES_ON=0
	for row in $NODE_LIST; do
		RES_ON=$(echo "$RES_CHECK" |grep "$row")
		if [ -n "$RES_ON" ]; then
			logger "$row online"
			((NODES_ON++))
		fi
	done
	if [ $NODES_ON -lt $NODES_N ]; then
		logger "Erro. Algum node nao esta online"
		exit
	fi

	#desativa serviços do node
	logger "Standby $NODE"
	/usr/sbin/crm node standby "$NODE"

	#aguarda node ficar livre, vms down e drbd down
	while :; do
		RES_CHECK_DRBD=$(/usr/sbin/crm resource show $STANDBY_CHECK_RES)
		RES_DRBD_ON=$(echo "$RES_CHECK_DRBD" |grep "$NODE")
		VMS_NUM=$(virsh list --name |wc -l)
	        if [ -z "$RES_DRBD_ON" ] && [ $VMS_NUM -le 1 ]; then
			logger "Pronto para reiniciar"
			break
		else
			logger "Servicos ainda executando"
		fi
		sleep 30
	done

	#escreve arquivo para recuperar node depois do reboot
	echo $(date +"%Y-%m-%d") > pacemaker_reboot.tmp

	#reinicia node
	logger "Rebooting..."
	/sbin/reboot

fi
\end{lstlisting}
