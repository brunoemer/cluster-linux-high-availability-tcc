\chapter{Alta disponibilidade}
Conceitualmente alta disponibilidade está diretamente relacionada a confiabilidade, dependabilidade e tolerância a falhas. 
Confiabilidade, a mais importante característica, transmite a ideia de continuidade de serviço \cite{pankaj1994}.
Por sua vez dependabilidade é o resultado de uma implementação de alta disponibilidade com sucesso em um determinado ambiente.

Alta disponibilidade é bastante conhecida atualmente, vem sendo cada vez mais empregada nos ambientes computacionais.
Pode-se defini-la como a redundância de \textit{hardware} ou \textit{software} para que o serviço fique mais tempo disponível.
De acordo com \cite{costa2009}, pode-se afirmar que alta disponibilidade está ligada á crescente dependecia de computadores.
O objetivo de promover alta disponibilidade resume-se em estar sempre a disposição quando cliente solicitar ou acessar algum serviço.

Uma das palavras-chave da alta disponibilidade é a tolerância a falhas.
Segundo \cite{pankaj1994}, o objetivo da tolerância a falhas é aumentar a disponibilidade de um sistema, isto é, 
aumentar o tempo de disponibilidade dos serviços oferecido aos usuários. Além disso um sistema é tolerante a falhas se ele pode
mascarar a presença de falhas em um sistem usando redundância. Como se expressa \cite{costa2009}, o objetivo da tolerância a falhas é 
alcançar a dependabilidade, assim indicando uma boa qualidade de serviço.

Redundância ...

Na prática se um componente falhar, ele deve ser reparado ou substituido por um novo.
