\chapter{Alta disponibilidade}
\label{cap:altadisponibilidade}

Alta disponibilidade é um termo muito conhecido, sendo cada vez mais empregada em ambientes computacionais. O objetivo de promover 
alta disponibilidade resume-se em garantir que um serviço esteja sempre disponível quando o cliente solicitar ou acessar \cite{costa2009}.
A alta disponibilidade é definida como a redundância de \textit{hardware} ou de \textit{software} para que o serviço fique mais tempo disponível.
Quanto maior for a disponibilidade desejada maior deverá ser a redundância no ambiente, assim reduzindo os pontos únicos de falha,
que em inglês são chamados de \ac{SPOF}. A alta disponibilidade está diretamente relacionada aos conceitos de: 
\begin{itemize}
 \item Dependabilidade: indica a qualidade do serviço fornecido e a confiança depositada neste serviço. A dependabilidade envolve atributos 
 como segurança de funcionamento, segurança de acesso, manutenabilidade, testabilidade e comprometimento do desempenho \cite{weber2002};
 \item Confiabilidade: é o atributo mais importante, pois transmite a ideia de continuidade de serviço \cite{pankaj1994}. A confiabilidade 
 refere-se a probabilidade de um serviço funcionar corretamente durante um dado intervalo de tempo;
 \item Disponibilidade: é a probabilidade de um serviço estar operacional no instante em que for solicitado \cite{costa2009};
 \item Tolerância a falhas: procura garantir a disponibilidade de um serviço utilizando mecanismos capazes de detectar, mascarar e recuperar 
 falhas, e seu objetivo é alcançar a dependabilidade, assim indicando uma boa qualidade de serviço \cite{costa2009}. A tolerância a falhas é 
 um dos principais conceitos da alta disponibilidade, sendo melhor discutida na Seção \ref{section:toleranciafalhas}.
\end{itemize}

\section{Tolerância a falhas}
\label{section:toleranciafalhas}

Sabe-se que o \textit{hardware} tende a falhar, principalmente devido a fatores físicos, por isso utiliza-se métodos para a prevenção 
de falhas e para a tolerância a falhas. A abordagem de prevenção de falhas define-se como um projeto feito na criação de mecanismos para
impedir que falhas ocorram. Além disso, a prevenção de falhas melhora a disponibilidade e a confiabilidade de um serviço, uma vez que essa 
tem como objetivo prever e eliminar o maior número de falhas possíveis antes de colocar o sistema em uso. 

A prevenção de falhas não resolverá todas as possíveis falhas. Sendo assim, a tolerância a falhas fornece disponibilidade de um serviço 
mesmo com a presença de falhas. Enquanto a prevenção de falhas tem foco nas fases de projeto, teste e validação, a tolerância a falhas 
tem como foco na utilização de componentes replicados para mascarar as falhas \cite{pankaj1994}.

O objetivo da tolerância a falhas é aumentar a disponibilidade de um sistema, ou seja, aumentar o intervalo de tempo que os serviços 
fornecidos ficam disponíveis aos usuários. Um sistema é dito tolerante a falhas se ele pode mascarar a presença de falhas ou recuperar-se 
de uma falha sem afetar o funcionamento do sistema. A tolerância a falhas é implementada utilizando redundância (Seção \ref{section:redundancia}). 
Um exemplo muito utilizado para tornar um sistema tolerante a falhas é a virtualização. Nestes ambientes normalmente existem dois servidores 
físicos nos quais máquinas virtuais estão sendo executadas, sendo que no caso de um dos servidores falhar, o \textit{software} de monitoramento fará a 
transferência das máquinas virtuais para o outro servidor, de forma transparente aos usuários, evitando assim a indisponibilidade do serviço. 
Este tema, virtualização, será detalhado no Capítulo \ref{cap:virtualizacao}.

A tolerância a falhas pode ser dividida em dois tipos que são: mascaramento; detecção, localização e reconfiguração.
O primeiro tipo, o mascaramento, não se manifesta na forma de erro sendo assim não necessita que o sistema trate este erro,
pois as falhas são tratadas na origem. O mascaramento é utilizado principalmente em sistemas de tempo real crítico. 
Um exemplo são os códigos de correção de erros, em inglês \ac{ECC}, que são utilizados em memórias para detecção e correção de erros.
O segundo tipo consiste em detectar, localizar a falha e reconfigurar o \textit{software} ou \textit{hardware} e por fim resolver a 
falha \cite{weber2002}. O segundo tipo de tolerância a falahas será decrito na Seção \ref{section:fasestolerancia}.

\subsection{Fases da tolerância a falhas}
\label{section:fasestolerancia}

A classificação das fases de tolerância a falhas mais comuns são \cite{weber2002}:

\begin{itemize}
 \item Detecção: realiza o monitoramento e aguarda uma falha se manifestar na forma de erro, para então passar para a próxima fase. 
 Um exemplo de detecção de erro é o cão de guarda (\textit{watchdog timer}), que recebe um sinal do programa ou serviço que esta sendo 
 monitorado e caso este sinal não seja recebido, devido alguma falha, o \textit{watchdog} irá se manifestar na forma de erro. 
 Um outro exemplo é o esquema de duplicação e comparação, onde são realizadas operações em componentes replicados com mesmos dados de 
 entrada, e então os dados de saída são comparados. No caso de diferenças nos dados de saída um erro é gerado.
 \item Confinamento: responsável pela restrição de um erro para que dados inválidos não se propaguem para todo o sistema, pois entre a falha e a
 detecção do erro há um intervalo de tempo. Neste intervalo pode ocorrer a propagação do erro para outros componentes do sistema, sendo assim 
 antes de executar medidas corretivas é necessário definir os limites da propagação. Na fase de projeto essas restrições devem ser previstas
 e tratadas. Um exemplo de confinamento é o isolamento de processos de um sistema operacional, onde existe restrições para que um processo
 não gere problemas em outro processo, com isso as falhas de um programa estão restritas ao processo. COMO É FEITO ???
 \item Recuperação: após a detecção de um erro ocorre a recuperação, onde o estado de erro é alterado para estado livre de erros. A recuperação
 pode ser feita de duas formas, que são:
 \begin{itemize}
  \item \textit{forward error recovery} (recuperação por avanço): ocorre uma condução para um novo estado não ocorrido anteriormente. É a forma
  de recuperação mais eficiente, porém mais complexo de ser implementado.
  \item \textit{backward error recovery} (recuperação por retorno): ocorre um retorno para um estado anterior que deve estar livre de erros.
  Para retornar ao estado anterior pode ser utilizados pontos de recuperação (\textit{checkpoints}), e quando ocorrer um erro, um 
  \textit{rollback} é executado, ou seja, o sistema retornará a um estado anterior a falha.
 \end{itemize}
 \item Tratamento: procura prevenir que futuros erros aconteçam. Nesta fase ocorre a localização da falha para descobrir o 
 componente que originou a falha. A substituição do componente danificado pode ser feita de forma manual ou automática. 
 O reparo manual é feito por um operador, e o automático quando existe um componente em espera para substituição. Exemplo de um reparo 
 manual é um operador que efetua a troca de um disco rígido de um servidor. E um exemplo de reparo automático é um disco configurado como 
 \textit{hot spare}, ou seja, um componente de \textit{backup} que assumirá o lugar do outro imediatamente após o componente principal 
 falhar. Em \textit{storages} ou servidores, o \textit{hot spare} pode ser configurado através de um \ac{RAID} \cite{rouse2013}.
\end{itemize}

\section{Redundância}
\label{section:redundancia}

A redundância pode ser implementada através da replicação de componentes, e tem como objetivo reduzir o número de \ac{SPOF} e garantir 
o mascaramento de falhas. Na prática, se um componente falhar ele deve ser reparado ou substituído por um novo, sem que haja uma 
interrupção no serviço. A redundância pode ser implementada ainda através do envio de sinais ou \textit{bits} de controle junto aos dados, 
servindo assim para detecção de erros e até para correção \cite{weber2002}. Segundo \cite{norvag2000} existem quatro tipos diferentes 
de redundância que são:
\begin{itemize}
 \item \textit{Hardware}: utiliza-se a replicação de componentes, sendo que caso um falhe outro possa assumir seu lugar. 
 Para fazer a detecção de erros a saída de cada componente é constantemente monitorada e comparada à saída do outro componente.
 Um exemplo prático de redundância de \textit{hardware} são os servidores com fontes redundantes. Nestes são utilizadas duas fontes ligadas 
 em paralelo, sendo que caso uma falhe a outra suprirá a necessidade de todo o servidor;
 \item Informação: ocorre quando uma informação extra é enviada ou armazenada para possibilitar a detecção e a correção de erros.
 Um exemplo são os \textit{checksums} (soma de verificação). Esses são calculados antes da transmissão ou armazenamento dos dados 
 e recalculados ao recebê-los ou recuperá-los, assim sendo possível verificar a integridade dos dados. Outro exemplo bastante comum são os 
 \textit{bits} de paridade que são utilizados para detectar falhas que afetam apenas um \textit{bit} \cite{weber2002};
 \item \textit{Software}: pode-se definir redundância de \textit{software} como a configuração de um serviço ou \textit{software} em
 dois ou mais locais diferentes. Pode-se citar um exemplo de um sistema gerenciador de banco de dados \textit{MySQL}, que pode ser configurado 
 com um modelo de replicação do tipo \textit{master-slave}, onde um servidor principal (\textit{master}) grava as operações em um arquivo, 
 para então os servidores \textit{slaves} recuperarem e executarem essas operações, com isso mantendo os dados sincronizados. Neste caso, 
 tanto o servidor \textit{master} quanto os \textit{slaves} executam o serviço \textit{MySQL} caracterizando assim uma redundância 
 \cite{viana201}. A redundância de \textit{software} também pode ser implementada com o objetivo de tolerar falhas e \textit{bugs} de um
 \textit{software} crítico, sendo que este \textit{software} é replicado em diferentes locais. Existem algumas técnicas que podem ser 
 utilizadas para essa implementação. Por exemplo, a programação de \textit{n}-versões, que consiste em criar \textit{n} versões para um 
 mesmo \textit{software}, desta forma, possibilita-se o aumento da disponibilidade, uma vez que elas provavelmente não apresentarão os 
 mesmos erros. Por outro lado a programação de \textit{n}-versões possui um custo muito elevado por isso não é muito utilizada.
 \item Tempo: este é feito através da execução de um conjunto de instruções repetidas vezes em um mesmo componente, assim detectando uma 
 falha caso ocorra. Essa técnica necessita tempo adicional, e é utilizada em sistemas onde o tempo não é crítico. Por exemplo, um 
 \textit{software} de monitoramento de serviços em servidores, que faz um teste em cada serviço e caso ocorra alguma falha, uma ação 
 corretiva  será executada para reestabelecer este serviço. Essa técnica, diferentemente da redundância de \textit{hardware}, não requer 
 um \textit{hardware} extra para sua implementação \cite{costa2009}.
\end{itemize}

\section{Cálculo da alta disponibilidade}

Um aspecto importante sobre alta disponibilidade é como medi-la. Para isso são utilizados os valores de \textit{uptime} e \textit{downtime}, 
que são respectivamente, o tempo que os serviços estão em execução e o tempo que não estão executando. A alta disponibilidade 
pode ser expressa pela quantidade de ``noves'', isto é, se um serviço possui quatro noves de disponibilidade, este possui uma 
disponibilidade de 99,99\% \cite{pereirafilho2004}.

\begin{table}
\caption {Níveis de alta disponibilidade e exemplos de sistemas}
\label{tab:dispniveis}
\begin{center}
\begin{tabular}{|l|l|l|l|}\hline
Nível & Uptime & Downtime por ano & Exemplos\\\hline
1 & 90\% & 36.5 dias & computadores pessoais\\\hline
2 & 98\% & 7.3 dias & \\\hline
3 & 99\% & 3.65 dias & sistemas de acesso\\\hline
4 & 99.8\% & 17 horas e 30 minutos & \\\hline
5 & 99.9\% & 8 horas e 45 minutos & provedores de acesso\\\hline
6 & 99.99\% & 52.5 minutos & CPD, sistemas de negócios\\\hline
7 & 99.999\% & 5.25 minutos & sistemas de telefonia ou bancários\\\hline
8 & 99.9999\% & 31.5 minutos & sistemas de defesa militar\\\hline
\end{tabular}
\end{center}
\end{table}

A Tabela \ref{tab:dispniveis} apresenta alguns níveis de disponibilidade, a suas porcentagens de \textit{Uptime}, os \textit{Downtime} por ano. 
A última coluna possui alguns exemplos de serviços relacionados ao nível de disponibilidade. Pode-se observar que para alguns serviços, 
como por exemplo, sistemas bancários ou sistemas militares é necessário um alto nível de disponibilidade \cite{pereirafilho2004}.

A porcentagem de disponibilidade (\textit{d}) pode ser calculada através da equação
\begin{equation}
d = \frac{MTBF}{(MTBF + MTTR)}
\label{disponibilidade}
\end{equation}
onde \ac{MTBF} é o tempo médio entre falhas, ou seja, corresponde ao tempo médio entre as paradas de um serviço. Já o \ac{MTTR} é o 
tempo médio de recuperação, isto é, o tempo entre a queda e a recuperação de um serviço \cite{goncalves2009}.

A alta disponibilidade é um dos principais fatores que fornece confiança aos clientes ou usuários, sendo extremante importante 
em empresas que fornecem serviços \textit{on-line}. Por isso, as empresas desenvolveram o \ac{SLA}, que é um acordo de nível de serviço, 
que garante que o serviço fornecido atenda as expectativas dos clientes. Um \ac{SLA} é um documento contendo uma descrição e uma definição 
das características mais importantes do serviço que será fornecido. Esse acordo também deverá conter a porcentagem de disponibilidade exigida
pelo negócio, sendo que esta deve ser minuciosamente definida. Por exemplo, um \ac{SLA} pode conter descrição do serviço, requerimentos, 
horário de funcionamento, entre outros \cite{smith2010}.
