\chapter{Alta disponibilidade}
Conceitualmente alta disponibilidade está diretamente relacionada a confiabilidade, dependabilidade e tolerância a falhas. 
Confiabilidade, a mais importante característica, transmite a ideia de continuidade de serviço \cite{pankaj1994}.
Por sua vez dependabilidade é o resultado de uma implementação de alta disponibilidade com sucesso em um determinado ambiente ou serviço,
sendo assim haverá uma dependência deste serviço.

Alta disponibilidade é bastante conhecida, vem sendo cada vez mais empregada nos ambientes computacionais.
Pode-se defini-la como a redundância de \textit{hardware} ou \textit{software} para que o serviço fique mais tempo disponível.
De acordo com \cite{costa2009}, pode-se afirmar que alta disponibilidade está ligada á crescente dependêcia de computadores.
O objetivo de promover alta disponibilidade resume-se em estar sempre a disposição quando o cliente solicitar ou acessar algum serviço.
Uma das palavras-chave da alta disponibilidade é a tolerância a falhas.

Sabe-se que \textit{hardware} tende a falhar por isso utiliza-se métodos como prevenção de falhas e tolerância a falhas.
A abordagem prevenção de falhas melhora a disponibilidade e a confiabilidade de um serviço porém, não resolverá todas as possíveis falhas.
Sendo assim, a segunda abordagem, tolerância a falhas, fornece disponibilidade mesmo com presença de falhas \cite{pankaj1994}.
O seu objetivo é aumentar a disponibilidade de um sistema, isto é, aumentar o tempo de disponibilidade dos serviços fornecidos aos clientes ou usuários. 
Um sistema é tolerante a falhas se ele pode mascarar a presença de falhas em um sistema usando redundância. Como se expressa \cite{costa2009}, 
o objetivo da tolerância a falhas é alcançar a dependabilidade, assim indicando uma boa qualidade de serviço.

Redundância pode ser geralmente feita através da replicação de componentes, para garantir o mascaramento de falhas. 
Também pode ser através do envio de sinais ou \textit{bits} de controle junto aos dados, servindo assim para detecção de erros e até para correção \cite{weber2002}.

Na prática se um componente falhar, ele deve ser reparado ou substituido por um novo...

Cálculo da diponibilidade
Disponibilidade = MTTF / (MTTF + MTTR)

Fazer: 
SPOF
SLA
MTTR
MTBF
