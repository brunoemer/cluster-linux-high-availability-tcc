\keyword{Alta disponibilidade}
\keyword{Virtualização}
\keyword{Tolerância a falhas}
\keyword{Código aberto}

\begin{abstract}

O número de serviços oferecidos através da Internet vem crescendo a cada ano que passa. Sendo assim, a alta disponibilidade é um fator
crucial para empresas que prestam este tipo de serviço. De fato, o aumento da disponibilidade é um grande diferencial para essas empresas.

Dentro deste contexto, neste trabalho é realizada uma implementação de um ambiente de alta disponibilidade em uma empresa prestadora 
de serviços para Internet, utilizando ferramentas de código aberto. Para isso, foi feita uma análise dos serviços oferecidos pela empresa, bem 
como um levantamento do ambiente de servidores da mesma.

Criou-se uma solução de alta disponibilidade, baseada no uso de um \textit{cluster} de computadores e virtualização. 
Para tal implementação foi necessário um \textit{software} para fazer a replicação de dados e outro para fazer o gerenciamento do 
\textit{cluster} e das máquinas virtuais. 
Alguns \textit{softwares} foram pesquisados e comparados para possibilitar a escolha do melhor e mais adequado para o ambiente da empresa.
%Logo, foram pesquisados alguns \textit{softwares}, tanto para replicação de dados como para gerenciamento de \textit{cluster}, 
%esses \textit{softwares} foram comparados e analisados para possibilitar a escolha dos melhores e mais adequados para o ambiente da empresa.
Mais especificamente foram utilizados os \textit{softwares} \textit{Pacemaker} para o gerenciamento do \textit{cluster}, e o \textit{software} 
\textit{DRBD}, que é responsável pela replicação dos dados.

Posteriormente, foram executados testes para simular falhas de \textit{hardware}, de energia elétrica e de \textit{software}, de forma a
validar o ambiente de alta disponibilidade criado. Os resultados mostraram que o tempo de indisponibilidade dos serviços neste ambiente de 
alta disponibilidade é consideravelmente menor se comparado ao antigo ambiente da empresa. 
Também mediu-se a disponibilidade dos serviços no antigo ambiente e no ambiente de alta disponibilidade criado, observando-se uma redução na 
indisponibilidade dos serviços.

\end{abstract}

\begin{englishabstract}{}{high availability, virtualization, fault tolerance, open source}

The number of services provided through the Internet has been growing each passing year. Thus, high availability is a crucial factor to the 
companies that provide this kind of service. In fact, the availability growth is a great differential for these companies.

Within this context, in this article an implementation of a high availability environment was accomplished in a company that provides 
Internet services, making use of open source tools. For this, an analysis of services provided by the company was done, 
as well as a survey regarding its servers environment.

A high availability solution was created, it was based on computer cluster and virtualization.
For such implementation a software for data replication and another for cluster management and for virtual machines was necessary. 
Some softwares were researched and compared to do the best and most adequate choice to the company environment.
Then, the softwares Pacemaker, which is a cluster manager, and the software DRBD that is the data replication software were used.

Lastly tests to simulate hardware fails, energy fails and software fails were executed, in order to validate the promoted high availability
environment. The results showed that the unavailability time of the services in the high availability environment is considerably less when 
compared to the old environment. Also, the services availability was measured in the old environment and in the high availability environment, 
observing an unavailability reduction in the services.

\end{englishabstract}
