\keyword{Alta disponibilidade}
\keyword{Virtualização}
\keyword{Tolerância a falhas}
\keyword{Código aberto}

\begin{abstract}

O número de serviços oferecidos através da Internet vem crescendo a cada dia que passa. Sendo assim, a alta disponibilidade é um fator
crucial para empresas que prestam este tipo de serviço. De fato, o aumento da disponibilidade é um grande diferencial para essas empresas.

Dentro deste contexto, neste trabalho é feito um estudo para a implementação de um ambiente de alta disponibilidade em uma empresa prestadora 
de serviços para Internet, utilizando ferramentas de código aberto. Para isso, foi feita uma análise dos serviços oferecidos pela empresa, bem 
como um levantamento do ambiente de servidores da mesma.

Criou-se um projeto de implementação para a solução de alta disponibilidade no ambiente da empresa, esse ambiente utiliza virtualização para uma 
grande parte dos serviços. Para tal, pesquisou-se algumas ferramentas de gerenciamento de \textit{cluster} e de replicação de dados. A partir 
destas ferramentas é feita uma análise das mesmas e adotada a mais adequada para o ambiente da empresa. Deste modo, o \textit{cluster} é composto 
pelo \textit{software} \textit{Pacemaker}, que faz o gerenciamento do \textit{cluster} e pelo \textit{software} \textit{DRBD}, que é responsável 
pela replicação dos dados.

Por fim validou-se o ambiente de alta disponibilidade através de testes que simulam falhas de \textit{hardware}, de energia elétrica e de
\textit{software}. Além disso, simulou-se manutenções e reinicializações do servidores físicos sem causar indisponibilidade dos serviços.

\end{abstract}

\begin{englishabstract}{}{high availability, virtualization, fault tolerance, open source}

The number of services provided through the Internet come growing each passed day. Thus, high availability is a crucial factor to the 
companies that provide this kind of service. In fact, the availability growth is a great differential for these companies.

Within this context, in this article was done a study to do a implementation of a high availability environment in a company that provide 
Internet services, making use at open source tools. For this, was done a analysis of services provide by the company, 
as well as a survey? its servers environment.

A implementation project was created for a high availability solution at company environment, this company uses virtualization to the biggest?
among? of its services. For such, was done a research some tools to cluster manage and to data replication?. With this tools was analyzed and 
adopted the most adequate? tool to the company environment. ?, the cluster was ? by software Pacemaker, that is the cluster manager, and by 
software DRBD, that is the data replication software.

? was validated? the high availability environment through at tests that simulate hardware fails, energy fails and software fails. 
In addition, was simulated maintenance and reboot of physical servers with cause unavailability of the services.

??

\end{englishabstract}
