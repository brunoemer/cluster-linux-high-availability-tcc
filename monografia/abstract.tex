\keyword{Alta disponibilidade}
\keyword{Virtualização}
\keyword{Tolerância a falhas}
\keyword{Código aberto}

\begin{abstract}

O número de serviços oferecidos através da Internet vem crescendo a cada dia que passa. Sendo assim, a alta disponibilidade é um fator
crucial para empresas que prestam este tipo de serviço. De fato, o aumento da disponibilidade é um grande diferencial para essas empresas.

Dentro deste contexto, neste trabalho é feito um estudo para a implementação de um ambiente de alta disponibilidade em uma empresa prestadora 
de serviços para Internet, utilizando ferramentas de código aberto. Para isso, foi feita uma análise dos serviços oferecidos pela empresa, bem 
como um levantamento do ambiente de servidores da mesma.

Criou-se um projeto de implementação para a solução de alta disponibilidade no ambiente da empresa, esse ambiente utiliza virtualização para uma 
grande parte dos serviços. Para tal, pesquisou-se algumas ferramentas de gerenciamento de \textit{cluster} e de replicação de dados. A partir 
destas ferramentas é feita uma análise das mesmas e adotada a mais adequada para o ambiente da empresa. Deste modo, o \textit{cluster} é composto 
pelo \textit{software} \textit{Pacemaker}, que faz o gerenciamento do \textit{cluster} e pelo \textit{software} \textit{DRBD}, que é responsável 
pela replicação dos dados.

Por fim validou-se o ambiente de alta disponibilidade através de testes que simulam falhas de \textit{hardware}, de energia elétrica e de
\textit{software}.

\end{abstract}

\begin{englishabstract}{}{high availability, virtualization, fault tolerance, open source}

??

\end{englishabstract}
