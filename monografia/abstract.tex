\keyword{Alta disponibilidade}
\keyword{Virtualização}
\keyword{Tolerância a falhas}
\keyword{Código aberto}

\begin{abstract}

O número de serviços oferecidos através da Internet vem crescendo a cada dia que passa. Sendo assim, a alta disponibilidade é um fator
crucial para empresas que prestam este tipo de serviço. De fato, o aumento da disponibilidade é um grande diferencial para essas empresas.

% Além do estudo sobre alta disponibilidade apresentado neste trabalho, será feito um estudo sobre virtualização, pois é um dos principais recursos 
% utilizados para se obter alta disponibilidade.

Dentro deste contexto, neste trabalho é feito um estudo para a implementação de um ambiente de alta disponibilidade em uma empresa prestadora 
de serviços para Internet, utilizando ferramentas de código aberto. Para isso, foi feita uma análise dos serviços oferecidos pela empresa, bem 
como um levantamento do ambiente de servidores da mesma.

Por fim, é elaborada uma proposta de solução de alta disponibilidade no ambiente da empresa, que utiliza virtualização para uma grande parte dos
serviços. Para tal proposta, pesquisou-se algumas ferramentas de gerenciamento de \textit{cluster} e de replicação de dados. A partir destas 
ferramentas é feita uma análise das mesmas e adotada a mais adequada para o ambiente da empresa. Deste modo, o \textit{cluster} é composto pelo 
\textit{software} \textit{Ganeti}, que faz o gerenciamento do \textit{cluster} e pelo \textit{software} \textit{DRBD}, que é responsável pela 
replicação dos dados.

% Além disso será pesquisado e testado ferramentas para possibilitar a criação de uma proposta e de uma implementação de alta disponibilidade 
% no ambiente atual da empresa.

\end{abstract}

\begin{englishabstract}{}{high availability, virtualization, fault tolerance, open source}



\end{englishabstract}
