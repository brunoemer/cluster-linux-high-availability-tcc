\keyword{Alta disponibilidade}
\keyword{Virtualização}
\keyword{Tolerância a falhas}
\keyword{Código aberto}

\begin{abstract}

O número de serviços oferecidos através da Internet vem crescendo a cada dia que passa. Sendo assim, a alta disponibilidade é um fator
crucial para empresas que prestam este tipo de serviço. De fato, o aumento da disponibilidade é um grande diferencial para essas empresas.

Dentro deste contexto, neste trabalho é realizada uma implementação de um ambiente de alta disponibilidade em uma empresa prestadora 
de serviços para Internet, utilizando ferramentas de código aberto. Para isso, foi feita uma análise dos serviços oferecidos pela empresa, bem 
como um levantamento do ambiente de servidores da mesma.

Criou-se um projeto de implementação para a solução de alta disponibilidade no ambiente da empresa, baseado no uso de um \textit{cluster}
de computadores e virtualização. Para tal implementação foi necessário um \textit{software} para fazer a replicação de dados e outro para fazer 
o gerenciamento e monitoramento do \textit{cluster}. Logo, foram pesquisados alguns \textit{softwares}, tanto para replicação de dados como para 
gerenciamento de \textit{cluster}, esses \textit{softwares} foram analisados e foram adotados os melhores e mais adequados \textit{softwares} 
para o ambiente da empresa.
Esses \textit{softwares} foram o \textit{Pacemaker}, que faz o gerenciamento do \textit{cluster}, e o \textit{DRBD}, que é responsável 
pela replicação dos dados.

Posteriormente, foram executados alguns testes para simular falhas de \textit{hardware}, de energia elétrica e de \textit{software}. Desta forma,
pôde-se validar o ambiente de alta disponibilidade e avaliar os resultados obtidos desses testes, os quais foram positivos. Esses resultados 
mostraram que o tempo de indisponibilidade dos serviços neste ambiente de alta disponibilidade é consideravelmente menor se comparado ao 
antigo ambiente da empresa. 
Também mediu-se a disponibilidade dos serviços no antigo ambiente e no ambiente de alta disponibilidade criado, comparou-se seus resultados e
pôde-se perceber a redução da indisponibilidade no recente ambiente de alta disponibilidade.

\end{abstract}

\begin{englishabstract}{}{high availability, virtualization, fault tolerance, open source}

The number of services provided through the Internet has come growing each passed day. Thus, high availability is a crucial factor to the 
companies that provide this kind of service. In fact, the availability growth is a great differential for these companies.

Within this context, in this article was accomplished a implementation of a high availability environment in a company that provide 
Internet services, making use at open source tools. For this, was done a analysis of services provided by the company, 
as well as a survey? its servers environment.

A implementation project was created for a high availability solution at company environment, it was based on computer cluster and virtualization.
For such implementation was necessary a software for data replication and other for cluster manage and monitoring. Then, was researched
some softwares both? data replication as cluster manager, it was analyzed and were adopted the better and most adequate tools to the company 
environment. These tools were the Pacemaker that is a cluster manager, and the DRBD that is the data replication software.

Lastly were executed some tests to simulate hardware fails, energy fails and software fails. So, it was possible validate the high availability
environment and evaluate these tests results, that was positive. Those results showed that the unavailability time of the services in the
high availability environment is considerably less when compared with the old environment. Also, the services availability was measured in
the old environment and in the high availability environment, so it was compared and was possible to see the unavailability reduction in the 
recent created high availability environment.

\end{englishabstract}
